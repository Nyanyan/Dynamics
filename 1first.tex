\chapter{はじめに}
これは私が趣味で力学を勉強するにあたって、理解するのに大事なところをまとめたものです。物理に興味がある高校生が読んで理解できる内容となるように書きました。また、必要な前提知識は十分丁寧に書いたつもりですので、意欲のある中学生にも理解していただける内容となっています。

力学に限らず物理はまず誰もが認めざるを得ない自然現象が存在します。そして、それをどうにかこうにか数式で(近似して)表していくのが「物理」的な視点です。

力学において最初に認める自然現象は以下のものです。

\begin{itemize}
    \item \bm{慣性の法則}\\
        物体は外部から力が加わらない限り、静止し続けるか等速直線運動をし続ける。
    \item \bm{運動方程式}\\
        物体に加速度が生じた時、その物体には加速度に比例し質量に反比例する力が加わった。
    \item \bm{作用反作用}\\
        2物体間で、一方が他方に力を加えると、力を加えた物体には他方の物体から同じだけの力が加わる。
    \item \bm{万有引力}\\
        質量のある2物体間には、各物体の質量にそれぞれ比例し、物体間の距離の2乗に反比例する引力が働く。
\end{itemize}

\clearpage