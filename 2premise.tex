\chapter{前提知識}
\label{premise}
物理は自然現象をどうにかこうにか数式で表そうとする学問ですから、相応の数学の知識が必要です。ここでは数学の知識を主軸に、本書を読む上で不可欠な前提知識を書いていきます。読者の方は知らないところだけを読むので問題ありません。なお、数学的知識の中でも最も大事な微分・積分については章を分けて解説します。

本章での説明はページ数の関係でかなり簡易的な説明です。読んでわからないところがありましたらご自身で調べてください。

\section{数学記号}
\label{math}
本書で頻出する数学記号とその意味、例を表\ref{tab:math}に示します。

\begin{table}[htb]
 \begin{center}
  \caption{数学記号とその意味}
  \label{tab:math}
  \begin{tabular}{c|l|l}
    \hline
    記号 & 意味 & 例・備考\\
    \hline\hline
    $\cap$ & かつ & A$\cap$B: AかつB\\
    $\in$ & 要素 & $a\in$B: $a$はBの要素である \\
    $\mathbb{C}$ & 複素数 & 複素数は実数$\cap$虚数\\
    $\mathbb{R}$ & 実数 & $\mathbb{C}\cap\overline{\mathbb{R}}$: 虚数\\
    $[a,b]$ & 区間$a\leq (変数)\leq b$ & 関数の区間を表すのによく使う\\
    
  \end{tabular}
 \end{center}
\end{table}



\section{ベクトルとスカラー}
\label{vector}
物理ではよくベクトル、スカラーという言葉が出てきます。スカラーはみなさんがよくご存知の、「ただの数」です。ではベクトルとは何なのでしょうか。簡単に説明すると、ベクトルは向きの情報を持ったスカラーです。

ベクトルとスカラーにはこのような違いがあるため、書き分けをする必要があります。本書では世間に一般に用いられている書き分け「ベクトルは太字」という方針で書きます。

なお、
\begin{equation}
    r = |\bm{r}|
\end{equation}

という書き方で、ベクトルの大きさ(スカラー)を表すことにします。



\subsection{内積}
ベクトルの演算で掛け算をしたいと思った時、掛け算の仕方には2通りあります。その一つが内積(スカラー積)です。スカラー積という名前の通り、演算の結果はスカラー量になります。

演算方法を2つ、記します。

\begin{eqnarray}
    \bm{a}\cdot\bm{b} &=& ab\cos\theta \\
    &=& a_1b_1+a_2b_2+a_3b_3+ ... + a_nb_n
\end{eqnarray}

なお、$\theta$は$\bm{a}$と$\bm{b}$のなす角、$a_1,a_2,...,b_1,b_2,...$はそれぞれ$\bm{a},\bm{b}$の成分です。

内積は、一方のベクトル$\bm{b}$をもう一方のベクトル$\bm{a}$に射影したとき\footnote{射影とは、$\bm{a}$と垂直な光線を$\bm{b}$に当てて、$\bm{b}$が$\bm{a}$に影を作るようにすることです。}の、影の長さ$b\cos\theta$と$\bm{a}$の長さ$a$の積です。

また、内積は演算方法からひと目で分かるように、$\bm{a}\cdot\bm{b}=\bm{b}\cdot\bm{a}$です。




\subsection{外積}
外積はクロス積とも言われ、また、3次元ベクトルでの外積は特にベクトル積と言います。物理では3次元ベクトルを主に使うので、3次元ベクトルで話を進めましょう。なお、演算結果はベクトルになります。

まずは演算方法です。

\begin{eqnarray}
    \bm{a}\times\bm{b}=
    \begin{pmatrix}
    a_2b_3-a_3b_2 \\
    a_3b_1-a_1b_3 \\
    a_1b_2-a_2b_1 \\
    \end{pmatrix}
\end{eqnarray}

出力されるベクトルは$\bm{a},\bm{b}$の両方に垂直で、掛けられた順番の右ねじの方向のベクトルとなります。このことからもわかるように、外積では$\bm{a}\times\bm{b}\neq\bm{b}\times\bm{a}$です。

なお、外積で求めるベクトルの大きさは、外積をとった2つのベクトルを2辺とした平行四辺形の大きさに等しくなります。




\section{色々な関数}
\label{function}
まず、関数とは「なにかを入力すると、なにかを返してくるもの」です。例えばお金を入れるとジュースが出てくる自動販売機は一種の関数です。また、関数は「函数」と書くこともあり、ちょうど自動販売機は箱の形をしているので、そのようなイメージを持っておくと良いでしょう。ですが、世間一般に関数という言葉は、「数値を入れると唯一つの数値を返してくるもの」としての使い方が多いと思います\footnote{余談ですが、プログラミング用語(と言って良いのかはわかりませんが)の「関数」は前者の「なにかを入力するとなにかを返してくるもの」としての意味合いが強く、入力には数値だけではなく、文字列だったり、何かの関数であったりします。ちなみにある関数に関数を入力するような状態を数学では合成関数と言います。微分・積分の範囲で物理でもよく使います。}。

物理現象を解明するために(かどうかの確証は持てませんが)、世の中には様々な関数があります。本書では、高校範囲で習わない関数についてはその都度説明を入れますが、高校範囲の関数については適宜説明はしません。わからない点がありましたら調べてみてください。

\if0
物理でよく用いる関数を列挙しましょう。

\begin{itemize}
    \item 三角関数\\
        sin(サイン)、cos(コサイン)、tan(タンジェント)などの表記を使う一連の関数。簡単な定義は三角形のそれぞれの辺の長さの比で、もう少し拡張した定義は単位円を用いた定義(図\ref{fig:trigonometric}参照)。
        \begin{figure}[!ht]
            \centering
            \includegraphics[width=12cm]{trigonometric.PNG}
            \caption{三角関数の定義}
            \label{fig:trigonometric}
        \end{figure}
        
\end{itemize}
\fi


\section{近似}
物理ではよく「近似式」というものを使います。よく使う近似には微小二乗項無視の近似と一次近似があります。また、一次近似とつながりのあるテイラー展開もここで説明します。


\subsection{微小二乗項無視}
物理では、「とても小さいものを2乗したらとてもとても小さいので無視して良いだろう」という考えをよく使います。具体的な例としては、$|x|\ll1$のとき、

\begin{eqnarray}
    (a+x)^2 \approx a^2 + 2ax \notag
\end{eqnarray}

等です。この式では$x^2$が微小二乗項として無視されました。




\subsection{一次近似}
例えば$\theta\ll1$の時、$\sin\theta \approx\theta$といったものです。これらは全て、一次近似という種類の近似です。

点$x=a$まわりでの一次近似の公式は以下です。$|x|\ll1$とします。

\begin{eqnarray}
    f(a+x) = f(a)+f'(a)x
\end{eqnarray}

ここで、$f'(a)$はグラフ$y=f(x)$の$x=a$での傾き(微分係数)を表します。微分について詳しくは\ref{differential}節を参照してください。

この式は、よく見ると点$(a,f(a))$での、$y=f(x)$の接線です。以下に代表的な近似を列挙しますが、これは一次近似でいつでも導けるものなので覚えなくて結構です。

\begin{table}[htb]
 \begin{center}
  \caption{一次近似の例($|x|\ll1$)}
  \label{tab:approx}
  \begin{tabular}{l|l}
    \hline
    元の式 & 近似式 \\
    \hline \hline
    $\sin x$ & $x$ \\
    $\cos x$ & $1$ \\
    $\tan x$ & $x$
  \end{tabular}
 \end{center}
\end{table}


\subsection{テイラー展開}
\label{taylor}
前節で一次近似を説明しました。では、この近似を永遠に続けていくとどうなるでしょうか。近似式であったはずの関数は限りなくもとの関数に近づきます。これをテイラー展開と言います。テイラー展開の公式を記しましょう。

\begin{eqnarray}
    f(x) &= \frac{f(a)}{0!}(x-a)^0 + \frac{f'(a)}{1!}(x-a)^1 + \frac{f''(a)}{2!}(x-a)^2 \notag \\
    &+ \frac{f^{(3)}(a)}{3!}(x-a)^3+... + \frac{f^{(n)}(a)}{n!}(x-a)^n + ...
\end{eqnarray}

また、特に$a=0$周りでのテイラー展開はマクローリン展開とも言い、以下の式で表せます。

\begin{eqnarray}
    f(x) &= \frac{f(0)}{0!}x^0 + \frac{f'(0)}{1!}x^1 + \frac{f''(0)}{2!}x^2 \notag \\
    &+ \frac{f^{(3)}(0)}{3!}x^3+... + \frac{f^{(n)}(0)}{n!}x^n + ...
\end{eqnarray}